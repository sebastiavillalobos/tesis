\resumenCastellano{

Como en cualquier tipo de competición, el objetivo principal es ganar y el videojuego de estrategia en tiempo real, StarCraft II, no es la excepción. Uno de los aspectos más importantes, en el ámbito competitivo de este juego, son los órdenes de construcción, que consisten en patrones de juego para guiar al jugador hasta una unidad o estructura deseada, cuya importancia radica en qué tan buenos son en contraste con los que utilicen los oponentes, dejando al jugador que posea el mejor orden de construcción con una ventaja en su estrategia respecto al contrincante. En esta investigación se detalla un modelo propuesto basado en la metaheurística \textit{Simulated Annealing} que permite obtener órdenes de construcción al nivel de los usados por profesionales del videojuego al optimizar el tiempo de partida que estas secuencias requieren para alcanzar una unidad, un edificio o una mejora de la raza Terran.

\vspace*{0.5cm}
\KeywordsES{StarCraft II, metaheurísticas, optimización, RCPSP, simulated annealing}
}

\newpage
