\glosario{
\label{glosario}
\begin{itemize}
\item \textbf{Árbol de tecnologías}: elemento del juego StarCraft II que muestra las unidades y estructuras necesarias para crear cada unidad y estructura de una raza. Se le atribuyen también las características de coste de tiempo de cada unidad y estructura. 
\item \textbf{Benchmark}: literalmente punto de referencia. Llámese de algo que es utilizado como punto comparativo. 
\item \textbf{Esports}: competiciones de videojuegos que conllevan eventos de gran escala. \item \textbf{Estructura}: uno de los elementos del juego StarCraft II que consiste de edificaciones o construcciones, cuyas principales funciones son las de creación de unidades e investigación de tecnologías. 
\item \textbf{Heurística}: se califica de heurístico a un procedimiento para el que se tiene un alto grado de confianza y que encuentra soluciones de alta calidad con un coste computacional razonable, aunque no se garantice su optimalidad o su factibilidad, e incluso, en algunos casos, no se llegue a establecer lo cerca que se está de dicha situación. 
\item \textbf{Metaheurística}: estrategias para diseñar y/o mejorar los procedimientos heurísticos orientados a obtener un alto rendimiento. 
\item \textbf{NP-Hard}: de las categorías de complejidad computacional, son aquellos problemas cuya resolución resulta más difícil que el problema más difícil de la categoría NP o Nondeterministic polynomial time. 
\item \textbf{Orden de construcción}: secuencia de acciones que indica que unidad o estructura debe ser creada en un punto de tiempo determinado, comenzando desde que inicia la partida, para lograr conseguir una unidad o estructura específica. Va de la mano con la estrategia que se desee utilizar. 
\item \textbf{Protoss}: una de las tres razas jugables del juego StarCraft II que posee unidades, estructuras y características propias en comparación al resto. Puede describirse como una raza robótica avanzada. 
\item \textbf{Ranking}: en el contexto de StarCraft II, una especie de tabla de posiciones que asigna rangos a los jugadores según su habilidad en partidas en línea clasificatorias. 
\item \textbf{Recursos}: uno de los elementos del juego StarCraft II que se representa por números enteros. Existen dos tipos de recursos: gas vespeno y minerales, los cuales deben ser extraídos desde el escenario para que vayan aumentando.
\item \textbf{Theorycrafting}: análisis matemático de las mecánicas de un juego con el fin de encontrar estrategias o tácticas. 
\item \textbf{Terran}: una de las tres razas jugables del juego StarCraft II que posee unidades, estructuras y características propias en comparación al resto. Puede describirse como una raza humana con tecnologías futuristas. 
\item \textbf{Unidad}: uno de los elementos del juego StarCraft II que consiste de un miembro del “ejército” de cada jugador y que puede movilizarse por el escenario. Aunque existen varias funciones dentro de las distintas unidades, su función principal es la de atacar. 
\item \textbf{Zerg}: una de las tres razas jugables del juego StarCraft II que posee unidades, estructuras y características propias en comparación al resto. Puede describirse como una raza de criaturas con forma de insectos.
\end{itemize}
}