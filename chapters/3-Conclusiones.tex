\chapter{Conclusiones}
\label{cap:conclusiones}

En el presente documento se explica cómo el videojuego StarCraft II tiene diversos trabajos científicos, con la mayoría apuntando a mejorar inteligencias artificiales que lo juega, dejando a un lado a los jugadores reales sin ofrecer soluciones para ayudarles a mejorar. Por esto es que se plantea un modelo que busca facilitar la búsqueda de los órdenes de construcción al utilizar un acercamiento metaheurístico que, tras representar los elementos necesarios del juego, logra ofrecer soluciones a la par con las que usan los jugadores profesionales de más alto nivel, demostrado por medio de experimentos.

En cuanto a los resultados obtenidos de los experimentos realizados en la sección 4, se logran optimizar los parámetros de entrada utilizados en el algoritmo SA. También se demuestra que las soluciones entregadas son realizables dentro del videojuego, lo cual refleja que las soluciones siguen las restricciones impuestas en este RCPSP. También se desprende que las soluciones entregadas están al nivel de las soluciones de profesionales, pues las diferencias entre los tiempos de juego de los órdenes de construcción ofrecidos son menores. Con esto en cuenta, el modelo aún tiene paso a cambios que permitan mejoras en las soluciones como la creación de múltiples entidades en un mismo segundo.

La pregunta de hipótesis (declarada en la sección 1.3) habla de conseguir buenos órdenes de construcción que fueron definidos como aquellos a la par o que superen los de jugadores profesionales o de alto rango clasificatorio, y los experimentos realizados junto a las comparaciones con órdenes de construcción de jugadores profesionales y de alto rango sugieren que los obtenidos a través del algoritmo implementado si pueden ser catalogados como buenos órdenes de construcción puesto que, en algunos casos, las soluciones entregadas por SA presentan claras ventajas, aunque sea en un par de segundos, sobre los órdenes de construcción con los que fueron contrastados.

Por otro lado, las soluciones que entrega el algoritmo son ejecutadas en un tiempo razonable si se contrastan con las horas que un jugador puede tardar para llegar a la misma respuesta o si, computacionalmente, se intentase obtener todas las posibles soluciones. Si bien el tiempo de ejecución para llegar a una solución no es útil para usarlo en tiempo real, este puede ser modificado para conseguir tiempos de ejecución menores al costo de disminuir la calidad de las soluciones.

También se puede decir que las soluciones consiguen llegar a la unidad requerida y en las cantidades establecidas, lo cual demuestra el último punto de la pregunta de hipótesis. Con esto, se permite afirmar que el modelo propuesto si puede encontrar órdenes de construcción de forma más eficiente que un orden de construcción aleatorio y en un tiempo razonable, el cual permite conseguir una unidad y una cantidad de esta determinadas, siempre que sea de la raza Terran en StarCraft II.

Los objetivos mencionados en la sección 1.4 fueron completados del todo, pues los órdenes de construcción obtenidos del modelo desarrollado si cumple con las características que se buscan de este en cuanto a la optimización del tiempo requerido en los órdenes de construcción, además de que el algoritmo se hace bajo un enfoque metaheurístico con el algoritmo SA.

En cuanto a los 4 objetivos específicos, se puede observar a lo largo del documento que estos fueron completados, pues tanto el objetivo 1 como el 2 y el 3 se presentan en el capítulo 3 que detalla el algoritmo propuesto y los experimentos a realizar sobre este. El objetivo 3 también se prueba junto al objetivo 4 en el capítulo 4 con los resultados de los experimentos planificados.

\section{Trabajo a futuro}

En vista de que las soluciones que ofrece el modelo tienen paso a cambios que permitan aumentar la eficiencia de las soluciones, existen algunos cambios que pueden mejorar la calidad de estas, como modificar el algoritmo para permitir más acciones por segundo (actualmente sólo permite una acción por segundo) o mejorar el cálculo de los recursos obtenidos por los trabajadores, ya que actualmente se utiliza un valor constante sin considerar la saturación que varía con la cantidad de trabajadores y tampoco considera el límite que tiene cada mineral o géiser de gas para ofrecer. Además de esto, se tiene la posibilidad de asignar un puntaje a cada entidad para que estas tengan un peso en la solución final, con el fin de escoger no sólo la solución con el tiempo más corto, sino que la solución que entregue un mayor beneficio al jugador en el menor tiempo posible, lo cual podría ayudar a acercarse o superar las soluciones usadas por profesionales que no sólo buscan construir una entidad en específico, sino que intentan conseguir un ejército capaz de superar al de su oponente.

Otro trabajo pendiente yace en las comparaciones, que si es necesario contrastarlas con otros órdenes de construcción ya existentes, no es necesario que estos sean de otros jugadores, pues también se pueden contrastar con soluciones entregadas con otras metaheurísticas para así analizar cual de ellas entrega mejores soluciones para este problema.

Si bien este trabajo se enfoca en la raza Terran de StarCraft II, el modelo puede ser expandido a las otras razas tras modelar sus características particulares que las diferencian de los Terran. Además, teniendo en cuenta el tipo de problema, el modelo puede aplicar no sólo a StarCraft II, sino que a cualquier videojuego de estrategia en tiempo real si se tienen en cuenta los elementos en común con este modelo enfocado en el RCPSP.

Para finalizar, se espera que tras incluir los puntos mencionados en esta sección, se pueda publicar ese nuevo documento a modo de aportar a la comunidad científica y a la comunidad de StarCraft.

\nocite{*}